% Created 2020-06-09 Tue 21:45
% Intended LaTeX compiler: pdflatex
\documentclass[11pt]{article}
\usepackage[utf8]{inputenc}
\usepackage[T1]{fontenc}
\usepackage{graphicx}
\usepackage{grffile}
\usepackage{longtable}
\usepackage{wrapfig}
\usepackage{rotating}
\usepackage[normalem]{ulem}
\usepackage{amsmath}
\usepackage{textcomp}
\usepackage{amssymb}
\usepackage{capt-of}
\usepackage{hyperref}
\usepackage{minted}
\usepackage{/home/ryan/Dropbox/profiles/Templates/LaTeX/ScreenStyle}
\author{Ryan Greenup}
\date{\today}
\title{Practice Exam 05 PCA and MDS Visualisation}
\hypersetup{
 pdfauthor={Ryan Greenup},
 pdftitle={Practice Exam 05 PCA and MDS Visualisation},
 pdfkeywords={},
 pdfsubject={},
 pdfcreator={Emacs 27.0.91 (Org mode 9.4)}, 
 pdflang={English}}
\begin{document}

\maketitle
\tableofcontents

\section{Trends 2014 Q8}
\label{sec:orge26256b}

Take the Following Data:

\begin{center}
\begin{tabular}{lrrr}
 & Period 1 & Period 2 & Period 3\\
Day 1 & 66 & 97 & 69\\
Day 2 & 68 & 111 & 87\\
Day 3 & 103 & 112 & 87\\
\end{tabular}
\end{center}

In order to examine the fluctuation for each day, break the data into its trend and periodic components, after a square root transformation:

\begin{minted}[]{r}
data <- rbind(
  c(66,  97,  69),
  c(68,  111, 87),
  c(103, 112, 87)
)
sqrt(data)
\end{minted}

\begin{verbatim}

          [,1]      [,2]     [,3]
[1,]  8.124038  9.848858 8.306624
[2,]  8.246211 10.535654 9.327379
[3,] 10.148892 10.583005 9.327379
\end{verbatim}



In this case the period is a time of day, so for example Morning, Lunch, Evening, so a seasonal trend would be expected.

Taking a moving average over the period of that trend however will remove the influence of that seasonality:

$$\begin{aligned}
Y  = T + S + Ɛ               \\
  &= T + 0 + 0               \\
  &= T
\end{aligned}$$

\subsection{(a) Calculate the Trend}
\label{sec:org4fa20ee}
Remember that the moving average the window expands evenly in both directions, so for example day2, period2 would be calculated as:

\begin{minted}[]{r}
library(tidyverse)
c(68, 11, 87)  %>%
  sqrt()  %>%
  sum() / 3
\end{minted}

\begin{verbatim}

[1] 6.963405
\end{verbatim}

\subsection{(b) Calculate the Periodic Component}
\label{sec:org0daf4cc}
The periodic component is S and can be calculated:


$$\begin{aligned}
Y  = T + S + Ɛ               \\
S = Y -T -Ɛ
\end{aligned}$$

but this doesn't sum to zero so instead:


$$\begin{aligned}
S = S' -1/m Σ S'
\end{aligned}$$

So the seasonal trend would be:

\begin{minted}[]{r}
library(zoo)
Y <- data

T <- rollmean(sqrt(as.vector(data)), k = 3)
Y_trim <- sqrt(head(as.vector(data), -1)[-1])
Sp <- Y_trim - T
S <- Sp - mean(Sp)

matrix(c("X", S, "X"), nrow = 3)
\end{minted}

\begin{verbatim}

     [,1]                 [,2]                 [,3]
[1,] "X"                  "-0.334760992717553" "-1.10486327843698"
[2,] "-0.599320243685816" "0.207330398617034"  "0.334433975456945"
[3,] "0.728420271117701"  "0.768759869648668"  "X"
\end{verbatim}

\subsubsection{Trend}
\label{sec:orgf765734}

\begin{center}
\begin{tabular}{lrrr}
 & Period 1 & Period 2 & Period 3\\
Day 1 & NA & 8.76 & 8.8\\
Day 2 & 9.03 & 6.96 & 10\\
Day 3 & 10.02 & 10.02 & NA\\
\end{tabular}
\end{center}


\subsubsection{Periodic}
\label{sec:org74de3d6}

Subtact the Trend from the observations:



S = Y - T


\begin{center}
\begin{tabular}{llll}
 & Period 1 & Period 2 & Period 3\\
Day 1 & NA & 8.76-sqrt(97) & 8.8-sqrt(69)\\
Day 2 & sqrt(68) - 9.03 & sqrt(111) - 6.96 & sqrt(87) - 10\\
Day 3 & sqrt(103) - 10.02 & sqrt(112) - 10.02 & NA\\
\end{tabular}
\end{center}


Now take the column averages

\begin{minted}[]{r}
sp <-
c(

mean(c(sqrt(68) - 9.03, sqrt(103) - 10.02)),

mean(c(sqrt(97) - 8.76,
 sqrt(111) - 6.96,
 sqrt(112) - 10.02)),


mean(c(sqrt(69) - 8.8,
  sqrt(87) - 10 ))

)
sp
\end{minted}

\begin{verbatim}

[1] -0.3274486  1.7425056 -0.5829985
\end{verbatim}


Unfourtunately this does not sum to zero so we fix that:

\begin{minted}[]{r}
(S <- sp - mean(sp))
\end{minted}

\begin{verbatim}
[1] -0.6048014  1.4651528 -0.8603514
\end{verbatim}


This does not match the sheet but I don't understand why?

\subsubsection{In practice}
\label{sec:org1f93cb7}
In practice we would just conclude that the values must sum to zero:

\begin{minted}[]{r}
p1 <- - 0.336
p3 <- - 0.594
# 0 = p1 + p2 + p3
(p2 = -p1 - p3)
\end{minted}

\begin{verbatim}

[1] 0.93
\end{verbatim}


Thus the missing componente is 0.93
\end{document}
